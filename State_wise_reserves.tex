\documentclass[12pt]{article}
\usepackage[pdfstartview=FitH,hidelinks]{hyperref}
\usepackage[british]{babel}
\usepackage{a4,graphicx}
\usepackage[a4paper, hmargin={2.05cm, 2.05cm}]{geometry} 
\usepackage{amsmath,amssymb,amsthm,mathtools}
\usepackage{anyfontsize}
\usepackage{bbm}
%\usepackage{xypic}
\usepackage[latin1,utf8]{inputenc}
\usepackage{marvosym}
\usepackage{etoolbox}
\usepackage{relsize}
\usepackage{needspace}
\usepackage{nameref}
\usepackage{dsfont}
%\usepackage{thmtools}
%\usepackage{ntheorem}
%\newtheorem{lho}{Sætning}
\usepackage{filecontents}
\usepackage{tikz}
\usetikzlibrary{arrows.meta,positioning,calc}
\usetikzlibrary{matrix}
\usepackage{empheq}

\newcommand*\widefbox[1]{\fbox{\hspace{2em}#1\hspace{2em}}}
\newcommand{\N}{\mathbb{N}}
\newcommand{\R}{\mathbb{R}}
\newcommand{\Q}{\mathbb{Q}}
\newcommand{\Z}{\mathbb{Z}}
\newcommand{\B}{\mathcal{B}}
\newcommand{\F}{\mathcal{F}}
\newcommand{\E}{\text{E}}
\newcommand{\cov}{\text{cov}}
\newcommand{\indic}[1]{\mathds{1}_{ \{ #1 \} }}
\newcommand{\unv}[1]{\mathds{1}_{  #1  }}
\newcommand\ddfrac[2]{\frac{\displaystyle #1}{\displaystyle #2}}
\newcommand{\noin}{\noindent}
\newcommand{\Var}{\text{Var}}
\renewcommand{\P}{\text{P}}
\renewcommand{\baselinestretch}{1.25} 

\newcommand\independent{\protect\mathpalette{\protect\independenT}{\perp}}
\def\independenT#1#2{\mathrel{\rlap{$#1#2$}\mkern2mu{#1#2}}}
\font\tt=rm-lmtl10

\newtheoremstyle{my_thm}% name
  {12pt}%         Space above, empty = `usual value'
  {12pt}%         Space below
  {\itshape}% Body font
  {}%         Indent amount (empty = no indent, \parindent = para indent)
  {\bfseries}% Thm head font
  {.}%        Punctuation after thm head
  {\newline}% Space after thm head: \newline = linebreak
  {}%         Thm head spec
\theoremstyle{my_thm}
\newtheorem{thm}{Theorem}[section]



\usepackage{caption}
\usepackage{listings,lstautogobble}
\usepackage{color}
\usepackage{float}
\usepackage{cprotect}
\usepackage[round, comma]{natbib}
\usepackage{csquotes}

\renewcommand{\mkbegdispquote}[2]{\itshape}

%\iffalse
\begin{filecontents*}{BIBS.bib}

@article{Djehiche,
issn = {01676687},
abstract = {We suggest a unified approach to claims reserving for life insurance policies with reserve-dependent payments driven by multi-state Markov chains. The associated prospective reserve is formulated as a recursive utility function using the framework of backward stochastic differential equations (BSDE). We show that the prospective reserve satisfies a nonlinear Thiele equation for Markovian BSDEs when the driver is a deterministic function of the reserve and the underlying Markov chain. Aggregation of prospective reserves for large and homogeneous insurance portfolios is considered through mean-field approximations. We show that the corresponding prospective reserve satisfies a BSDE of mean-field type and derive the associated nonlinear Thiele equation. [web URL: http://www.sciencedirect.com/science/article/pii/S0167668715300548]},
journal = {Insurance, Mathematics and Economics},
volume = {69},
publisher = {Elsevier Sequoia S.A.},
year = {2016},
title = {Nonlinear reserving in life insurance: Aggregation and mean-field approximation},
language = {eng},
address = {Amsterdam},
author = {Djehiche, Boualem and Löfdahl, Björn},
keywords = {Studies ; Life Insurance ; Insurance Claims ; Insurance Policies ; Differential Equations ; Stochastic Models ; Life and Health Insurance ; Experiment/Theoretical Treatment},
url = {http://search.proquest.com/docview/1806433652/},
}

@article{Norberg,
journal = {Scand. Actuar. J. 1},
year = {1991},
title = {Reserves in Life and Pension Insurance},
author = {Norberg, Ragnar},
pages={3-24}}

@book{Pardoux,
series = {Stochastic Modelling and Applied Probability},
volume = {69},
publisher = {Springer International Publishing},
isbn = {9783319057132},
year = {2014},
title = {Stochastic Differential Equations, Backward SDEs, Partial Differential Equations},
edition = {2014},
language = {eng},
address = {Cham},
author = {Pardoux, Etienne and Rascanu, Aurel},
keywords = {Mathematics ; Probability Theory and Stochastic Processes ; Partial Differential Equations ; Mathematics},
}

@article{THM_BUC,
issn = {03461238},
abstract = {Within the setup of a semi-Markov process in a finite state space, we consider a life insurance contract. First, without the modelling of policyholder behaviour, we show how to calculate the expected cash flow associated with future payments, and to that end we present a version of Kolmogorov's forward integro-differential equation. The semi-Markov model is then extended to include modelling of surrender and free policy behaviour, and the main result is a modification of Kolmogorov's forward integro-differential equation, such that the cash flow can be calculated without significantly more complexity than the cash flow without policyholder modelling. The result is also demonstrated for the traditional Markov case where there is no duration dependence, and numerical examples are studied.},
journal = {Scandinavian Actuarial Journal},
volume = {2015},
publisher = {Taylor and Francis Ltd.},
number = {8},
year = {2015},
title = {Cash flows and policyholder behaviour in the semi-Markov life insurance setup},
language = {eng},
address = {Stockholm},
author = {Buchardt, Kristian and Møller, Thomas and Schmidt, Kristian},
keywords = {Studies ; Cash Flow ; Markov Analysis ; Life Insurance ; Differential Equations ; Experiment/Theoretical Treatment ; Management Science/Operations Research ; Life and Health Insurance},
url = {http://search.proquest.com/docview/1718994569/},
}



@book{Liv2,
series = {International Series on Actuarial Science},
publisher = {Cambridge University Press},
isbn = {0521868777},
year = {2007},
title = {Market-valuation methods in life and pension insurance},
language = {eng},
address = {Cambridge},
author = {Møller, Thomas and Steffensen, Mogens},
keywords = {Insurance - Mathematics; Insurance, Life - Policies - Mathematics; Insurance, Pension trust guaranty - Mathematics; Life insurance policies - Mathematics; Pension trust guaranty insurance - Mathematics; Pension trusts - Mathematics; Økonomi, forsikring},
}

@article{Steffensen0,
issn = {01676687},
abstract = {The multi-state life insurance contract is reconsidered in a framework of securitization where insurance claims may be priced by the principle of no arbitrage. This way a generalized version of Thiele's differential equation is obtained for insurance contracts linked to indices, possibly marketed securities. The equation is exemplified by a traditional policy, a simple unit-linked policy and a half-dependent unit-linked policy.},
journal = {Insurance, Mathematics and Economics},
pages = {201--214},
volume = {27},
publisher = {Elsevier Sequoia S.A.},
number = {2},
year = {2000},
title = {A no arbitrage approach to Thiele's differential equation},
language = {eng},
address = {Amsterdam},
author = {Steffensen, Mogens},
keywords = {Arbitrage ; Securitization ; Market Prices ; Insurance Policies ; Stochastic Models ; Economic Theory ; Studies ; Economic Theory ; Insurance Industry ; Experimental/Theoretical},
url = {http://search.proquest.com/docview/208166412/},
}

@book{Steffensen1,
publisher = {Laboratory of Actuarial Mathematics, University of Copenhagen},
isbn = {8778344492},
year = {2001},
title = {On valuation and control in life and pension insurance},
language = {eng},
address = {Copenhagen},
author = {Steffensen, Mogens},
}



@article{NorbergB,
issn = {0949-2984},
abstract = {The issue of bonus in life insurance is considered in a model framework where the traditional set-up is extended by letting the experience basis (mortality, interest, etc.) be stochastic. A novel definition of the technical surplus on an insurance contract is proposed, and basic principles for its repayment as bonus are discussed. Making the experience basis an endogenous part of the model opens possibilities of model-based prognostication of future bonuses. Numerical illustrations are provided.},
journal = {Finance and Stochastics},
pages = {373--390},
volume = {3},
publisher = {Springer-Verlag},
number = {4},
year = {1999},
title = {A theory of bonus in life insurance},
language = {eng},
address = {Berlin Heidelberg},
author = {Norberg, Ragnar},
keywords = {Key words: Safety margins, prospective reserves, retrospective reserves, stochastic interest, stochastic mortality, counting processes ; JEL Classification:G22, G23 ; Mathematics Subject Classification (1991):60J27, 62P05},
}


@article{Christiansen,
issn = {0167-6687},
journal = {Insurance Mathematics and Economics},
pages = {132--137},
volume = {57},
publisher = {ELSEVIER SCIENCE BV},
number = {1},
year = {2014},
title = {Reserve-dependent benefits and costs in life and health insurance contracts},
language = {English},
author = {Christiansen, MC and Denuit, MM and Dhaene, J},
keywords = {Insurance ; Contracts ; Cost-Benefit Analysis ; Economic Analysis ; Economics;},
}




\end{filecontents*}

%\fi



\begin{document}

\subsection*{Stuff to fix}
\begin{itemize}
\item Jeg kommer med flere påstande om industrien som jeg ikke er sikker på har hold i virkeligheden.
\item Der er i princippet ikke nogen grund til at vi regner retrospektivt, når vi alligevel ikke bruger historikken... Skal vi udvidde, så X kan afhænge lineært af tidligere værdier? Vi kræver blot at
$$
\E[g(t,Z(t),\{ X(\tau) \}_{\tau\leq t})|Z(t)=i]=g(t,i,\{ \E[X(\tau)|Z(t)=i] \}_{\tau \leq t})$$
fx ved
$$
g(t,i,\{X(\tau)\}_{\tau \leq t})=\int_0^t f^1_i(\tau) X(\tau) d\nu_1(\tau,t) + \int_0^t f^2_i(\tau) X(\tau) d\nu_2(\tau,t)
$$
for ét eller andet sigma-additivt mål $\nu_1$ (og $\nu_2$). Fx kunne $\nu_1$ være lebesque målet fra $t-1$ til $t$, mens $\nu_2$ kunne være punktmålet i $t$ hvilket er specialtilfældet som vi i øjeblikket kigger på. Vi kan også lade $g$ afhænge af tidligere værdier af $Z$, fx på følgende måde
$$
g(t,\{Z(\tau)\}_{\tau\leq t},\{X(\tau)\}_{\tau\leq t})=
\int_0^t f(Z(\tau),\tau,t) X(\tau) d\nu(\tau),
$$
hvorved
\begin{align*}
&\E_{Z(0)}[\indic{Z(r)=i}g(t,\{Z(\tau)\}_{\tau\leq t},\{X(\tau)\}_{\tau\leq t})|Z(t-)=g]
\\
=&
p_{gi}(t,r)\int_0^t \E[ f(Z(\tau),\tau,t) X(\tau)|Z(t-)=g] d\nu(\tau,t)
\\
=&
\int_0^t p_{gi}(t,r) \E \left[ \sum_{j \in \mathcal{J}}  \indic{Z(\tau)=j} f(j,\tau,t) \E[X(\tau)|Z(\tau)=j,Z(t-)=g] \bigg| Z(t-)=g \right] d\nu(\tau,t)
\\
=&
\int_0^t p_{gi}(t,r) \E \left[ \sum_{j \in \mathcal{J}}  \indic{Z(\tau)=j} f(j,\tau,t) \frac{\E[X(\tau)\indic{Z(\tau)=j}]}{p_{0j}(0,\tau)} \bigg| Z(t-)=g \right] d\nu(\tau,t)
\\
=&
\int_0^t p_{gi}(t,r) \sum_{j \in \mathcal{J}}  P(Z(\tau)=j|Z(0)=0,Z(t)=g) f(j,\tau,t) \frac{\tilde{X}^j(\tau)}{p_{0j}(0,\tau)}  d\nu(\tau,t)
\\
=&
\int_0^t \sum_{j \in \mathcal{J}} \frac{p_{jg}(\tau,t)}{p_{0g}(0,t)} p_{gi}(t,r) f(j,\tau,t) \tilde{X}^j(\tau) d\nu(\tau,t).
\end{align*}
hvor vi har brugt at $X(\tau)|Z(\tau)$ er uafhængig af $Z(t)|Z(\tau)$ for $\tau\leq t$. På denne måde kunne man fx. lade dividenden være en klumpbetaling svarende til det gennemsnitlige forventede bidrag over det sidste år, altså $f(j,\tau,t)X(\tau)=X(\tau)(r(\tau)-r^*(\tau))\sum_{k\neq j} \rho_1^{jk}(t)+\sum_{k\neq j} \rho_2^{jk}(t)$ for $\nu(\tau,t)$ værende lebesque målet for $(t-1,t]$ hvis $t$ er et heltal, og ellers 0. Da $\nu$ ikke nødvendigvis er absolut kontinuert, vil $g$ ikke svare til den kontinuerte udvikling af $X$ - vi tillader klump-betalinger på deterministiske tidspunkter.
\item Tilsvarende for $h$
$$
h(t,\{Z(\tau)\}_{\tau\leq t},\{X(\tau)\}_{\tau\leq t})
=\int_{(0,t]} \sum_{k \neq Z(\tau-)} \phi(\tau,t,Z(\tau-),k) X(\tau-) dN^k\otimes\nu(\tau,t)
$$
Taking the expectation and conditioning on $Z(t-)=g$
\begin{align*}
&\E[\indic{Z(r)=j} h(t,\{Z(\tau)\}_{\tau\leq t},\{X(\tau-)\}_{\tau\leq t})|Z(t-)=g]
\\
=&
\int_{(0,t]} \E \left[ \sum_{k \neq Z(\tau-)} \phi(\tau,t,Z(\tau-),k) \indic{Z(r)=j} X(\tau-) dN^k\otimes\nu(\tau,t) \bigg| Z(t-)=g \right]
\\
=&
\int_{(0,t]} \E \left[ \E \left[ \sum_{k \neq Z(\tau-)} \phi(\tau,t,Z(\tau-),k) \indic{Z(r)=j}X(\tau-) dN^k\otimes\nu(\tau,t) |Z(\tau-), Z(t-)=g \right] \bigg| Z(t-)=g \right]
\\
=&
\int_{(0,t]} \E \left[ \sum_{i \in \mathcal{J}}\indic{Z(\tau)=i} \E \left[ \sum_{k \neq i} \phi(\tau,t,i,k) \indic{Z(r)=j}X(\tau-) dN^k\otimes\nu(\tau,t) |Z(\tau-)=i, Z(t-)=g \right] \bigg| Z(t-)=g \right]
\\
=&
\int_{(0,t]} \E \left[ \sum_{i \in \mathcal{J}}\indic{Z(\tau)=i} \sum_{k \neq i} \phi(\tau,t,i,k) \E \left[  \indic{Z(r)=j}X(\tau-) dN^k\otimes\nu(\tau,t) |Z(\tau-)=i, Z(t-)=g \right] \bigg| Z(t-)=g \right]
\\
=&
\int_{(0,t]} \E \left[ \sum_{i \in \mathcal{J}}\indic{Z(\tau)=i} \sum_{k \neq i} \phi(\tau,t,i,k) \E \left[  \indic{Z(r)=j}X(\tau-) dN^k(\tau) |Z(\tau-)=i \right]\nu(\tau,t) \bigg| Z(t-)=g \right]
\intertext{og da $X(\tau-)|Z(\tau-)$ er uafhænig af $\indic{Z(r)=j}dN^h(\tau)|Z(\tau-)$}
=&
\int_{(0,t]} \E \left[ \sum_{i \in \mathcal{J}}\indic{Z(\tau)=i} \sum_{k \neq i} \phi(\tau,t,i,k) \E \left[ X(\tau-) |Z(\tau-)=i \right] \E[ \indic{Z(r)=j} dN^k(\tau) |Z(\tau-)=i]\nu(\tau,t) \bigg| Z(t-)=g \right]
\\
=&
\int_{(0,t]} \E \left[ \sum_{i \in \mathcal{J}}\indic{Z(\tau)=i} \sum_{k \neq i} \phi(\tau,t,i,k) \frac{\tilde{X}^i(\tau)}{p_{0i}(0,\tau)} \E[dN^k(\tau) |Z(\tau-)=i,Z(r)=j]p_{ij}(\tau,r)\nu(\tau,t) \bigg| Z(t-)=g \right]
\\
=&
\int_{(0,t]}  \sum_{i \in \mathcal{J}} \E \left[\indic{Z(\tau)=i}  | Z(t-)=g \right] \sum_{k \neq i} \phi(\tau,t,i,k) \frac{\tilde{X}^i(\tau)}{p_{0i}(0,\tau)} \mu_{ik|ij}(\tau|\tau,r) p_{ij}(\tau,r) d\nu(\tau,t)
\\
=&
\int_{(0,t]}  \sum_{i \in \mathcal{J}} \frac{p_{0i}(0,\tau)p_{ig}(\tau,t)}{p_{0g}(0,t)} \sum_{k \neq i} \phi(\tau,t,i,k) \frac{\tilde{X}^i(\tau)}{p_{0i}(0,\tau)} \mu_{ik|ij}(\tau|\tau,r) p_{ij}(\tau,r) d\nu(\tau,t)
\\
=&
\int_{(0,t]}  \sum_{i \in \mathcal{J}} \frac{p_{0i}(0,\tau)p_{ig}(\tau,t)}{p_{0g}(0,t)} \sum_{k \neq i} \phi(\tau,t,i,k) \frac{\tilde{X}^i(\tau)}{p_{0i}(0,\tau)} \mu_{ik}(\tau)\frac{p_{kj}(\tau,r)}{p_{ij}(\tau,r)} p_{ij}(\tau,r) d\nu(\tau,t)
\\
=&
\int_{(0,t]}  \sum_{i \in \mathcal{J}} \frac{p_{ig}(\tau,t)}{p_{0g}(0,t)} \sum_{k \neq i} \phi(\tau,t,i,k)\tilde{X}^i(\tau) \mu_{ik}(\tau)p_{kj}(\tau,r) d\nu(\tau,t)
\end{align*}

\end{itemize}
\section{Introduction}
With-profit insurance contracts are to this day one of the most popular life insurance contracts. They arose as a natural way to distribute the systematic surplus that develops due to the prudent assumptions on which the contract is made. In recent years, sensible questions accompanied by a lot of attention have been aimed at the surplus, to name a few; is it distributed fairly? what is the risk carried by the equity? how should it be invested? One might look for answers in the existing literature e.g. \citet{Liv2}, \citet{NorbergB}, \citet{Steffensen0} or \citet{Steffensen1}, where partial differential equations are used to describe the prospective second order reserve and its interplay with surplus. While these PDE's provide a conceptually powerful tool, they are limited to simple market dynamics, and they do not provide realistic models for long term financial markets. \citet{NorbergB} considers the development of the surplus in a financial Markov chain environment, allowing for great model flexibility but even still, an important element is completely neglected: the human element.
\\[12pt]
Insurance companies are governed by humans, and the decisions they make have an influence on the portfolio of policies - in particular concerning surplus and dividends. In a with-profit insurance contract many quantities are fixed at initialisation of the policy, but the rate at which dividends are accrued is not. The insurance company has a certain degree of freedom when it comes to the distribution of surplus, and the actions that have an influence on the insurance contracts are the so-called Management Actions.
%As stated by CEIPOS\footnote{https://eiopa.europa.eu/CEIOPS-Archive/Documents/Advices/CEIOPS-L2-Final-Advice-TP-Assumptions-future-management-actions.pdf} \begin{displayquote} The actuarial and statistical methods used to calculate the best estimate should take account of the effect on these future cash-flows of potential future actions by the management of (re)insurance undertakings based upon current and credible information. \end{displayquote}
From a mathematical point of view, they pose a problem as they depend on the entire history of the portfolio of policies, making it difficult to calculate prospective reserves. If we want to take a glance into the crystal ball of liabilities, taking Future Management Actions (FMA's) into account, we need to embrace it's retrospective nature.
\\[12pt]
In this paper we derive a retrospective differential equation for the expected savings account and surplus, in a general model with affine dynamics.  


\subsection{Prospective vs. Retrospective}
When incorporating human decisions into the projection of balances and benefits in life insurance, we need to embrace the fact that these decisions are based on the past. How to embrace the retrospective nature, is the main contribution of this paper.\\

\begin{itemize}
\item Something about Monte-carlo method
\item Something about FMA's - perhaps an example?
\item Deterministic second order basis, and discussion regarding simulation.
\end{itemize}

\subsection{Set-up}
We consider the classic multi-state life insurance set-up, comprised of a state process $Z$ denoting the state of the policy in a finite state space $\mathcal{J}=\{0,1,...,J\}$. The counting process $N^k$ defined by $N^k(t)=\# \{ s; Z(s-)\neq k, Z(s)=k, s \in (0,t] \}$ describes the number of transitions into state $k$. The benefits less premiums, for a certain contract $i$, develops in accordance with
$$
dB_i(t)=b_i^{Z(t)}(t) dt +\sum_{k \neq Z(t-)} b_i^{Z(t-)k}(t)dN^k(t),
$$
where $b_i^j(t)$ and $b_i^{jk}(t)$ are deterministic payment functions specifying payments during sojourns in state $j$ and on transition from state $j$ to state $k$, respectively. Even though single payments during sojourns in states pose no mathematical difficulty, we assume that payments during sojourns in states are continuous for notational simplicity. The state process $Z$ is assumed to be a continuous time Markov chain, with transition probabilities denoted by
$$
p_{ij}(s,t)= \P(Z(t)=j|Z(s)=i)
$$
for $s\leq t$. The corresponding transition intensities are denoted by
$$
\mu_{ij}(t)=\lim_{h \searrow 0} p_{ij}(t,t+h)/h
$$
for $i \neq j$. 
\subsubsection{Savings and Surplus}
The savings $X$, and surplus $Y$, are the two quantities of interest. In practice, the dynamics of these accounts are very specific, and not a subject of debate. However, instead of working with the explicit dynamics of $X$ and $Y$, we choose to use the more general affine dynamics given by
\begin{align*}
dX(t)=&X(t)g_{x1}(t,Z(t),Y(t))dt + \sum_{k \neq Z(t-)} X(t-) h_{x1}(t,Z(t-),k,Y(t-)) dN^k(t)\\
&+g_{x2}(t,Z(t),Y(t))dt + \sum_{k \neq Z(t-)} h_{x2}(t,Z(t-),k,Y(t-)) dN^k(t),
\\
dY(t)=&Y(t)g_{y1}(t,Z(t),X(t))dt + \sum_{k \neq Z(t-)} Y(t-) h_{y1}(t,Z(t-),k,X(t-)) dN^k(t)
\\
&+g_{y2}(t,Z(t),X(t))dt + \sum_{k \neq Z(t-)} h_{y2}(t,Z(t-),k,X(t-)) dN^k(t).
\end{align*}
While it may not seem so, we work with $g$ and $h$ functions mainly for notational reasons. We refer to section \ref{seq:Dyn} for a practically relevant choice of the functions $g$ and $h$ - the notational advantage of using $g$ and $h$ is also apparent there. It is important to realize the extent of applicable models that have affine dynamics, see \citet{Christiansen} for several relevant payment functions that are linear in the reserve, which corresponds to the savings when $D(t)=0$ for all $t$. While the reach of models with affine dynamics is extensive, there are limitations to consider. It is not uncommon to have dynamics that include some min or max function, for instance in the case of guarantees, and these non-linear functions in savings cannot be described by affine dynamics. \\
As stated in the introduction, management actions are one of the main motivators of this paper, but where are they evident in the dynamics of the savings and surplus? The answer is, the management actions are not evident. They are hidden in mainly two terms, the second order interest and the dividend. This is because the management decides how to invest the surplus, and how it should be distributed to the customers. Due to the very human and abstract nature of management actions, we do not incorporate them directly in the dynamics of the savings and surplus, but instead let them work in the shadows. It is important to 


\begin{itemize}
\item Even though FMA's are one of the main reasons for considering the savings account, they are hidden in the dividend and surplus investment strategy.
\end{itemize}

Let $W$ be some possibly multidimensional process with $Z$-dependent dynamics
\begin{align*}
dW(s)=& g(s,Z(s),W(s)) ds+ \sum_{k\neq Z(s-)} h(s,Z(s-),k,W(s-)) dN^k(s),
\end{align*}
for $g$ and $h$ functions that are linear in all elements of $W$. This multidemsional process can for instance represent the savings and surplus
$$
W(s)= \begin{pmatrix}
X(s)\\
Y(s)
\end{pmatrix}.
$$
In practice, the surplus account is shared among policyholders, corresponding to $W\in \R^{N+1}$ for $N$ policies with a state process $Z$ on a state space of size $\# \{ \mathcal{J} \}^N$; one for each combination of all policy states. There are several ways to reduce the dimensionality of the problem, making it computationally tractable.

\subsection{One Active State}
We consider a simple model where the expected future savings are described by an easily derived differential equation. The model consists of $n$ inactive states where there are no payments, and one active state with continuous dynamics $g$ which, in this setting, may be non-linear. Denote by $0$ the active state. On transition to any one of the inactive states, the surplus and savings are nullified. We need not specify what happens to the surplus and savings on a transition - they may be paid out to the customer or the insurance company, or any combination of the two - the only important requirement is that they are zero in all inactive states. The eradication of surplus and savings on transition corresponds to the relation $h_x(t,0,j,x,y)+h_y(t,0,j,x,y)=-x-y$, for $j=1,...,n$. The survival model with and without surrender options are special cases of this model.  The dynamics of $X$ and $Y$ are
\begin{align*}
dX(s)=& \indic{Z(s-)=0} g_x(s,0,X(s),Y(s))ds - \sum_{h=1}^n X(s-)dN^h(s)
\\
dY(s)=& \indic{Z(s-)=0} g_y(s,0,X(s),Y(s))ds - \sum_{h=1}^n Y(s-)dN^h(s).
\end{align*}
%Note that there are no risk premiums, as there is no risk for the insurance company company related to the transitions of the policy. Therefore, the only risk carried by the insurance company, relates to the interest of the savings account.
Let $W(s)=(X(s),Y(s))^T$, and denote by $T_1$ the time of the first jump. For the deterministic function $W_a$ that solves
$$
W_a(t)=\int_0^t g(s,0,W_a(s)) ds,
$$
we see that
$$
\hat{W}(t):=\E[W(t)|Z(0)=0] = \E [  \indic{t<T_1} W_a(t)|Z(0)=0]  = p_{00}(0,t) W_a(t),
$$
which comes at no surprise. In this case we know the past and present values of $W$ given the current state of $Z$, so the only stochastic element pertains to the state of the policy at time $t$. By differentiating w.r.t. $t$, and applying Kolmogorov's forward differential equation, we get the following forward differential equation for $\hat{W}$,
\begin{align*}
\hat{W}(0)=&\begin{pmatrix}
X(0)\\
Y(0)
\end{pmatrix},
\\
\frac{d}{dt}\hat{W}(t)=&p_{00}(0,t) g \left( s,0,\frac{\hat{W}(t)}{p_{00}(0,t)}\right)
-
\frac{\hat{W}(t)}{p_{00}(0,t)}\sum_{k=1}^n \mu_{0k}(t).
\end{align*}
Even though it may seem very simple and perhaps even trivial, the model with one active state has great applicability.

\subsubsection{Example With One Active State}
If the benefits are identical after age 65, the states 0,1,3 and 4 can be lumped, as well as 2,5 and 6, thus creating a survival model. If the dynamics in two states are identical, they can be viewed as one. Life annuity at age 65.
\def\PlA{(0,0)}
\begin{figure}[H]
\begin{center}
\begin{tikzpicture}
\begin{scope}[every node/.style={rectangle,thick,draw,inner sep=10pt,minimum width=2cm,minimum height = 1cm},rounded corners=1mm]
    \node (0) at \PlA {0, alive};
    \node (1) at ($(0) + (4.5,0)$) {1, dead};
\end{scope}

\begin{scope}[>={Stealth[black]},
              every node/.style={fill=white,circle,scale=0.9},
              every edge/.style={draw=black,very thick}]
    \path [->] 	(0) edge [bend left=0] node {$\mu$} (1);
\end{scope}
\end{tikzpicture}
\label{fig:1}
\caption{Life-Death model}
\end{center}
\end{figure}


\subsection{Two Active States}
When expanding to a model where there are two active states, and $n$ inactive states, we need to use a different method to calculate $\hat{W}$, \textit{if the active states are transient}. There is an important difference between the hierarchical model with two active states, and the transient model with two active states. In the model with one active state, we know the entire history of the policy, given that the policy is in the active state. When we introduce a second active state in the hierarchical model, we also know where the policy has been given the active state, but we do not know when it transitioned from one active state to the other. In order to calculate the expectation of the savings and surplus, we simply have to integrate over all possible transition times. If there are two transient states, there is an infinite amount of paths to any of the transient states, and for each possible path there is an infinite amount of possible jump times.
%Due to the large amount of possible paths and jump times, the naïve approach of calculating $\hat{W}$ is intractable. However, if the dynamics are affine, we can cut some very large corners.
To illustrate the naïve method of calculating expected savings and surplus in a hierarchical model, consider the model depicted in figure \ref{fig:2}
 \def\PlA{(0,0)}
\begin{figure}[H]
\begin{center}
\begin{tikzpicture}
\begin{scope}[every node/.style={rectangle,thick,draw,inner sep=10pt,minimum width=2cm,minimum height = 1cm},rounded corners=1mm]
    \node (0) at \PlA {0, active};
    \node (1) at ($(0.center) + (4.9,0)$) {1, active};
    \node (2) at ($(0.center) + (0,-3)$) {inactive};    
    \node (3) at ($(0.center) + (4.9,-3)$) {inactive};    
\end{scope}

\begin{scope}[>={Stealth[black]},
              every node/.style={fill=white,circle,scale=0.9},
              every edge/.style={draw=black,very thick}]
    \path [->] 	(0) edge [bend left=0] node {$\mu_{01}$} (1);
    \path [->] 	(0) edge [bend left=0] (2);
    \path [->] 	(1) edge [bend left=0] (3);
\end{scope}
\end{tikzpicture}
\caption{Two active state hierarchical model}
\label{fig:2}
\end{center}
\end{figure}
\noin In this model, there are two states for which the savings and surplus are non-zero; $Z(t)\in \{ 0, 1\}$. As in the case with one active state, we know the value of $W(t)$ for $Z(t)=0$, but for $Z(t)=1$ we need to consider all possible transition times. Let $T_1$ be the time of the transition from 0 to 1. If $W_0$ solves
\begin{align*}
W_0(t)=&\int_0^t g(s,0,W_0(s)) ds,
\intertext{then it characterizes the expected value of $W(t)$, given that $Z(t)=0$. Similarly, $W_1$ characterizes the expected value of $W(t)$ given $Z(t)=1$ and $dN_{01}(T_1)=1$, if it solves}
W_1(T_1,T_1)=&W_0(T_1)+h(T_1,0,1,W_0(T_1)), \\
W_1(T_1,t)=&\int_{T_1}^t g(s,1,W_1(T_1,s)) ds.\\
\intertext{The density of $T_1$, given that $Z(t)=1$ is}
q(s,t)=&\frac{p_{00}(0,s)p_{11}(s,t)}{p_{01}(0,t)}\mu_{01}(s).
\end{align*}
Let $T_1$ be the time of the first jump, then
\begin{align*}
W(t)= \indic{Z(t)=0} W_0(t)+\indic{Z(t)=1}W_1(T_1,t),
\end{align*}
implying that
\begin{align*}
\E [ W(t)] =& p_{00}(0,t) \E[W_0(t)|Z(t)=0] + p_{01}(0,t) \E[W_1(T_1,t)|Z(t)=1].
\end{align*}
%Note that
%\begin{align*}
%\E[W(t)|Z(t)=i]=
%\begin{dcases}
%W_0(t)\quad &\text{ for } Z(t)=0\\
%\int_0^t q(s,t) W_1(s,t) ds &\text{ for } Z(t)=1 \\
%0 &\text{ otherwise }. \\
%\end{dcases}
%\end{align*}
When $Z(t)=0$ all information about the history of the policy is known, and the value of $W$ is deterministic. Conditioning on $Z(t)=1$ does not provide full information about the history of the policy, as we do not know the time at which the transition from state 0 to state 1 was made. Therefore, to calculate $\E[W(t)|Z(t)=1]$ we have to integrate over all possible transition times, weighted by the transition intensity given that a jump happened prior to $t$. Thus
$$
\E [ W(t)]=p_{00}(0,t) W_0(t) + p_{01}(0,t) \int_0^t q(s,t) W_1(s,t) ds.
$$
We could apply this method of calculating $\E[W(t)]$ to any model. The basic principle is simple: given all information about the past of $Z$, we can calculate the value of $W(t)$, and the expected past can be calculated for each possible path of the policy. In general $\hat{W}$ can be calculated as
\begin{align}
\E[W(t)]= \sum_{i \in \mathcal{P}}  P(\text{path }i) \int_{(0,t]^{L_i}} W_i(t,\Theta_{L_i}) dP_i(\Theta_{L_i}),
\label{eq:AAA}
\end{align} 
where $\mathcal{P}$ is the set of possible policy paths, $L_i$ is the length of path $i$, $\Theta_{L_i}$ is an $L_i$-dimensional vector of jump-times, $dP_i$ is the density of transition times for path $i$ and $W_i(t,\Theta_{L_i})$ is the value of $W(t)$ given the path and transition times.
\\
When the model is small and hierarchical, \eqref{eq:AAA} provides a tractable method to calculate the expected savings and surplus, as there are few possible paths and they are short. When the model is transient the problem explodes, as there are infinitely many paths for the policy to take. Fortunately, there are some very large corners to cut, under the simple assumption that $g$ and $h$ are affine in $W$. Consider the case where $W$ has dynamics
$$
dW(s)=g(s,Z(s))W(s) ds,
$$
then
\begin{align*}
W(t)
%=& \int_0^{T_1} g(s,0)W(s) ds + \int_{T_1}^t g(s,1) W(s) ds \\
%=& \int_0^t \indic{Z(s)=0} g(s,0)W(s) + \indic{Z(s)=1} g(s,1) W(s) ds
%\\
=& \int_0^t W(s) g(s,Z(s)) ds.
\end{align*}
Say we want to calculate 
$$
\tilde{W}^i(t):=\E_{Z(0)}[W(t)\indic{Z(t)=i}]
%=\E_{Z(0)}[W(t)|Z(t)=i]p_{Z(0)i}(0,t),
$$ 
as we can use it to calculate $\E[W(t)]=\tilde{W}^0(t)+\tilde{W}^1(t)$. By the tower property and Fubinis theorem,
\begin{align*}
\E_{Z(0)}[W(t)\indic{Z(t)=1}]
=&
\int_0^t \E[ \indic{Z(t)=1} W(s) g(s,Z(s))] ds
\\
=&
\int_0^t \E_{Z(0)}[ \indic{Z(s)=1}\E_{Z(0)}[ \indic{Z(t)=1} W(s) g(s,Z(s))|Z(s)=1]] ds\\
&+
\int_0^t \E_{Z(0)}[ \indic{Z(s)=0}\E_{Z(0)}[ \indic{Z(t)=1} W(s) g(s,Z(s))|Z(s)=0]] ds
\\
=&
\int_0^t p_{Z(0)1}(0,s) \E_{Z(0)}[ \indic{Z(t)=1} W(s) |Z(s)=1] g(s,1) ds\\
&+
\int_0^t p_{Z(0)0}(0,s)\E_{Z(0)}[ \indic{Z(t)=1} W(s) |Z(s)=0] g(s,0)ds.
\intertext{By the Markov property $W(s)\independent Z(t)|Z(s)$, as $W(s)$ is $\mathcal{F}_s$-measurable.}
=&
\int_0^t p_{11}(s,t)\tilde{W}^1(s) g(s,1) ds\\
&+
\int_0^t p_{01}(s,t) \tilde{W}^0(s) g(s,0)ds.
\end{align*}
Differentiating with respect to $t$, and using Kolmogorov's forward differential equations yields the following system of differential equations
\begin{align*}
\frac{d}{dt} \tilde{W}^1(t)=& \mu_{01}(t)\tilde{W}^0(t)(t)-\mu_{10}(t)\tilde{W}^1(t)
+ \tilde{W}^1(t)g(t,1)
\\
\frac{d}{dt} \tilde{W}^0(t)=& \mu_{10}(t)\tilde{W}^1(t)(t)-\mu_{01}(t)\tilde{W}^0(t)
+ \tilde{W}^0(t)g(t,0)
\\
\tilde{W}^i(0)=&\indic{Z(0)=i}W(0).
\end{align*}
It is crucial to note that this differential equation is invariant to whether or not the model is transient, in contrast to the naive approach where all possible paths need to be considered individually. In the following section we generalise the result, 

\section{State-Wise Probability Weighted Reserve}
In the previous section we presented a differential equation for a simple two state model, without any payments on transition. The same methodology can be applied for a general Markov model with affine dynamics. We are interested in $\tilde{W}^i(t)$ for $i \in \mathcal{J}$, noting by the tower property that the relation between $\tilde{W}^i$ and $\E_{Z(0)}[W(t)]$ is given by
\begin{align*}
\E_{Z(0)}[W(t)] =&
% \E_{Z(0)}[\E_{Z(0)} [ X(t)|Z(t)]] 
%\\
%=&
%\E_{Z(0)} \left[ \sum_{j\in \mathcal{J}} \indic{Z(t)=j} \E_{Z(0)} [ X(t)|Z(t)=j] \right]
%\\
%=&
\E_{Z(0)} \left[ \sum_{i\in \mathcal{J}} \indic{Z(t)=i} \frac{\E_{Z(0)}[W(t)\indic{Z(t)=i}]}{p_{0i}(0,t)} \right]
\\
%=&
%\sum_{j\in \mathcal{J}} p_{0j}(0,t) \frac{ \tilde{X}^j(t)}{p_{0j}(0,t)}
%\\
=&
\sum_{i\in \mathcal{J}} \tilde{W}^i(t).
\end{align*}
By using the tower property and the fact that $W(s-)\independent Z(t)|Z(s-)$, we get the following theorem
\begin{thm}[]
\label{thm:Diff_1}
Let $Z(t)$ be a Markov process on the state space $\mathcal{J}$, and let $W(t)$ be a $\mathcal{F}_t$ measurable process with dynamics
\begin{align*}
dW(s)=  g(s,Z(s),W(s))ds+
 \sum_{k \neq Z(s-)} h(s,Z(s-),k,W(s-)) dN^k(s) 
\end{align*}
for $g$ and $h$ of the form
\begin{align*}
g(s,Z(s),W(s))=&g_1(s,Z(s)) W(s)+g_2(s,Z(s))
\\
h(s,Z(s-),k,W(s-))=&h_1(s,Z(s-),k) W(s-)+h_2(s,Z(s),k)
\end{align*}
for some measures $\nu_g,\nu_h,\eta_g$ and $\eta_h$.
Then
\begin{align*}
\frac{d}{dt}\tilde{W}^i(t)=&
\sum_{j \neq i} \mu_{ji}(t) \tilde{W}^j(t)-\mu_{ij}(t)\tilde{W}^i(t)
\\
&+
\tilde{W}^i(t)g_1(t,i)+p_{Z(0)i}(0,t)g_2(s,Z(s))
\\
&+
\sum_{j\neq i} \mu_{ji}(t) \left( \tilde{W}^j(t) h_1(t,j,i)+ p_{Z(0)j}(0,t)h_2(t,j,i)\right)
\end{align*}
\end{thm}

\begin{proof}[Proof of theorem \ref{thm:Diff_1}]
By the tower property
\begin{align*}
\tilde{W}^i(t):=&\E_{Z(0)}[W(t) \indic{Z(t)=i}]
\\
=&
\E_{Z(0)} \left[ \int_0^t \indic{Z(t)=i} dW(s) \right]
\\
=&
\E_{Z(0)} \left[ \int_0^t \indic{Z(t)=i} g(s,Z(s),W(s))ds \right]
\\
&+
\E_{Z(0)} \left[ \int_0^t \sum_{k \neq Z(s-)} \indic{Z(t)=i} h(s,Z(s-),k,W(s-)) dN^k(s)  \right].
\end{align*}
Note that $h(s,Z(s-),k,W(s-)$ is predictable, and we may therefore replace the integrator $dN^k(s)$ with its predictable compensator $\indic{Z(s-)\neq k}\mu^{Z(s-)k}(t)$. Using the tower property
\begin{align*}
\tilde{W}^i(t)=&
\int_0^t \E_{Z(0)} \left[ \E_{Z(0)} \left[ \indic{Z(t)=i} g(s,Z(s),W(s))|Z(s) \right]\right] ds
\\
&+
 \int_0^t\E_{Z(0)} \left[ \E_{Z(0)} \left[ \sum_{k \neq Z(s-)}\indic{Z(t)=i} h(s,Z(s-),k,W(s-))  \mu^{Z(s-)k}(t) |Z(s-) \right] \right] ds
\\
=&
\int_0^t \sum_{j \in \mathcal{J}} p_{0j}(0,s) \E_{Z(0)} \left[ \indic{Z(t)=i} g(s,Z(s),W(s))|Z(s)=j\right] ds
\\
&+
 \int_0^t \sum_{j \in \mathcal{J}} p_{0j}(0,s)  \E_{Z(0)} \left[ \sum_{k \neq Z(s-)} \indic{Z(t)=i} h(s,Z(s-),k,W(s-))  \mu^{Z(s-)k}(t) |Z(s-)=j \right] ds
 \\
=&
\int_0^t \sum_{j \in \mathcal{J}} p_{0j}(0,s)\left( \E_{Z(0)} \left[ \indic{Z(t)=i} W(s))|Z(s)=j\right]g_1(j,s)+g_2(j,s)\right) ds
\\
&+
\int_0^t \sum_{j \in \mathcal{J}} p_{0j}(0,s)  \left( \sum_{k \neq j}  \mu^{jk}(t) \left(  \E_{Z(0)} \left[\indic{Z(t)=i} W(s-)|Z(s-)=j \right] h_1(s,j,k)+h_2(s,j,k)  \right) \right) ds
\end{align*}
By the Markov property
$$
\E_{Z(0)}[\indic{Z(t)=i}W(s)|Z(s)=j]=\frac{\tilde{W}^j(t)}{p_{Z(0)j}(0,s)}p_{ji}(s,t)
$$
\end{proof}

The functions $g$ and $h$ respectively describe the continuous and discrete dynamics of $W(s)$. Skriv noget om antagelser og signifikans. Eksempel.
\subsubsection{Free Policy}

\subsection{General Path dependent dynamics}
Even though theorem \ref{thm:Diff_1} provides a powerful too for calculating future values of the savings and surplus, it is restricted to dynamics that only depends on the current value of $Z$ and $W$. It is not unreasonable to assume that the dividend strategy depends on the history of $Z$ and $W$. It turns out, that for certain dynamics that, in some sense, are linearly dependent on the past we can 


\begin{thm}[]
\label{thm:Diff_2}
Let $Z(t)$ be a Markov process on the state space $\mathcal{J}$, and let $W(t)$ be a $\mathcal{F}_t$ measurable process with dynamics
\begin{align*}
dW(s)= d g(s,\{Z(\tau)\}_{\tau\leq s},\{W(\tau)\}_{\tau\leq s})+
 d h(s,\{Z(\tau)\}_{\tau\leq s},\{W(\tau)\}_{\tau\leq s})
\end{align*}
for $g$ and $h$ of the form
\begin{align*}
g(s,\{Z(\tau)\}_{\tau\leq s},\{W(\tau)\}_{\tau\leq s})=&\int_{(0,s]} \varphi_1(s,\tau,Z(\tau))W(\tau) d\nu_g(\tau,t)
\\
&+
\int_{(0,s]} \varphi_2(s,\tau,Z(\tau)) d\eta_g(\tau,t)
\\
h(s,\{Z(\tau)\}_{\tau\leq s},\{W(\tau)\}_{\tau\leq s})=&\int_{(0,s]} \sum_{k\neq Z(\tau-)} \psi_1(s,\tau,Z(\tau-),k) W(\tau-)  dN^k\otimes\nu_h(\tau,t)
\\
&+
\int_{(0,s]} \sum_{k\neq Z(\tau-)} \psi_2(s,\tau,Z(\tau-),k)  dN^k\otimes\eta_h(\tau,t),
\end{align*}
for some measures $\nu_g,\nu_h,\eta_g$ and $\eta_h$.
Then
\begin{align*}
\frac{d}{dt}\tilde{W}^i(t)=&
\sum_{j \neq i} \mu_{ji}(t) \tilde{W}^j(t)-\mu_{ij}(t)\tilde{W}^i(t)
\\
&+
\int_{(0,t]} \sum_{k \in \mathcal{J}} p_{ki}(\tau,t) \varphi_1(t,\tau,k) \tilde{W}^k(\tau) d\nu_g(\tau,t) ds
\\
&+
p_{Z(0)i}(0,t)\int_{(0,t]} \sum_{k \in \mathcal{J}} p_{ki}(\tau,t) \varphi_2(t,\tau,k) d\eta_g(\tau,t) ds
\\
\tilde{W}^i(0)=&\indic{Z(0)=i}W(0)
\end{align*}

\end{thm}
We see that theorem \ref{thm:Diff_1} is a special case of theorem \ref{thm:Diff_2} with $\nu_g,\nu_h,\eta_g$ and $\eta_h$ being the Dirac measures in $t$, i.e 
$$
\nu_g(\tau,t)=\nu_h(\tau,t)=\eta_g(\tau,t)=\eta_h(\tau,t)=\indic{\tau=t}
$$



\subsection{Life-Death-Surrender}
\def\PlA{(0,0)}
\begin{center}
\begin{tikzpicture}
\begin{scope}[every node/.style={rectangle,thick,draw,inner sep=10pt,minimum width=2cm,minimum height = 1cm},rounded corners=1mm]
    \node (0) at \PlA {0, alive};
    \node (1) at ($(0.center) + (4.9,0)$) {1, dead};
    \node (2) at ($(0.center) + (-4.5,0)$) {2, surrender};    
\end{scope}

\begin{scope}[>={Stealth[black]},
              every node/.style={fill=white,circle,scale=0.9},
              every edge/.style={draw=black,very thick}]
    \path [->] 	(0) edge [bend left=0] node {$\mu^{01}$} (1);
    \path [->] 	(0) edge [bend left=0] node {$\mu^{02}$} (2);
\end{scope}
\end{tikzpicture}
\end{center}



\subsection{Life-Death-Surrender With Free Policy}
\def\PlA{(0,0)}
\def\PlB{(0,-3)}
\begin{center}
\begin{tikzpicture}
\begin{scope}[every node/.style={rectangle,thick,draw,inner sep=10pt,minimum width=2cm,minimum height = 1cm},rounded corners=1mm,align=center]
    \node (0) at \PlA {0, alive};
    \node (1) at ($(0.center) + (4.5,0)$) {1, dead};
    \node (2) at ($(0.center) + (-4.5,0)$) {2, surrender}; 
    \node (3) at \PlB {3, alive\\ free policy};
    \node (4) at ($(3.center) + (4.5,0)$) {4, dead \\ free policy};
    \node (5) at ($(3.center) + (-4.5,0)$) {5, surrender \\ free policy};    
\end{scope}

\begin{scope}[>={Stealth[black]},
              every node/.style={fill=white,circle,scale=0.9},
              every edge/.style={draw=black,very thick}]
    \path [->] 	(0) edge [bend left=0] node {$\mu^{01}$} (1);
    \path [->] 	(0) edge [bend left=0] node {$\mu^{02}$} (2);
    \path [->] 	(0) edge [bend left=0] node {$\mu^{03}$} (3);
    \path [->] 	(3) edge [bend left=0] node {$\mu^{34}$} (4);
    \path [->] 	(3) edge [bend left=0] node {$\mu^{35	}$} (5);
\end{scope}
\end{tikzpicture}
\end{center}
\subsection{Use of Savings account}

\subsection{Thoughts}
\begin{itemize}
\item With-profit insurance! Expected reserve including accumulation of dividends.
\item Refer to \citet{Norberg}
\begin{itemize}
\item Introduction and motivation - stochastic reserve, Monte Carlo method. A little comment on the fact that the problem is still hard to solve.
\item Life-death (simple analytic solution).
\item Life-death free policy (how to deal with extra states).
\item General model without duration.
\item Life-death-surrender free policy, including discussion of free policy factor.
\item Lost all trick works.
\item General model with duration dependence.
\item Inclusion of surplus. Use independence when dividend is assigned on discrete points in time.
\end{itemize}
\item Deterministic intensities.
\item General Hierarchical models do not need linearity. In general the variance increases as the number of states increase as the variance of the sum of transition times increases.
\item Market dependent intensities - allowed when directly dependent on the market, making them deterministic. Or intensities that depend on the expected reserve - in a sense corresponding to intensities that depend on the group of similar policies.
\item We are only concerned with the reserve.
\item Maybe we should use a different wording? \textbf{Savings}/stash/backlog/accumulation/hoard/reservoir instead of reserve, to distinguish between the Danish words for "reserve" and "depot"
\item One could imagine that information about the jump time could be partially deduced from the intensities, thus almost allowing for non-linearity. Consider case where $\mu_{01}(t)= \kappa \indic{t \in (c_1,c_2]}$ for very small $|c_2-c_1|$ and very large $\kappa$, providing almost perfect information about the jump time, whereby non-linearity in $g(s,1,W(s))$ would be allowed for.

\end{itemize}



\appendix

\section{Proof of Theorem \ref{thm:Diff_1} }
\section{Dynamics of $X$ and $Y$}\label{seq:Dyn}
The savings account at time $t$ is the past net income compounded with the real interest, i.e.
$$
X(t):=\int_0^t e^{\int_s^t r} d(B(s)+D(s)-C(s)),
$$
while the surplus at time $t$ is the past contributions less dividends compounded with the real interest, i.e.
$$
Y(t):=\int_0^t e^{\int_s^t r} d(C(s)-D(s))
$$


The amount by which the savings surpass the first order reserve, is spent on $B_2$. 
where
\begin{gather*}
b^j(t,x)=b_1^j(t)+\frac{x-V_1^{j*}(t)}{V_2^{j*}(t)}b^j_2(t),
\qquad \quad
b^{jg}(t,x)=b_1^{jg}(t)+\frac{x-V_1^{j*}(t)}{V_2^{j*}(t)}b^{jg}_2(t).
\end{gather*}
Dynamics of $X$
\begin{align*}
dX(t)=&
r^*(t)X(t)dt
 +\delta^{Z(t)}(t,X(t),Y(t))  dt- \sum_{g \neq Z(t-)} \rho^{Z(t-)g}(t,X(t-)) dt
 \nonumber 
\\
\nonumber
&- b^{Z(t)}(t,X(t)) dt
\\
&- \sum_{g\neq Z(t-)}\bigg(b^{Z(t-)g}(t,X(t-))+\chi^{Z(t-)g}(t,X(t-))-X(t-) \bigg) \mu^{Z(t)g}(t)dt
\\
&+ \sum_{g\neq Z(t-)}\bigg(\chi^{Z(t-)g}(t,X(t-))-X(t-) \bigg)  dN^g(t),
\end{align*}
and
\begin{align*}
dY(t)=Y(t)\frac{dS(t)}{S(t)}-\delta^{Z(t)}(t,X(t),Y(t)) + (r(t)-r^*(t)) X(t) + \sum_{g \neq Z(t-)} \rho^{Z(t)g}(t,X(t)),
\end{align*}
where
\begin{align*}
\rho^{jg}(t,x)=&(b^{jg}(t,x)+ \chi^{jg}(t,x)-x) (\mu^{*jg}(t)-\mu^{jg}(t))
\\
\chi^{jg}(t,x)=& V^{g*}_1(t) + \frac{x-V^{j*}_1(t)}{V^{j*}_2(t)}V^{g*}_2(t),
\end{align*}
\begin{equation}
\delta^j(t,x,y)=\delta_1^j(t)+\delta_2^j(t)x+\delta_3^j(t)y+\delta_4^j(t)xy.
\end{equation}

\begin{align*}
\tilde{W}^j(t):=
\begin{pmatrix}
\tilde{X}^j(t) \\
\tilde{Y}^j(t)
\end{pmatrix}
=
\begin{pmatrix}
\E[ X(t) \indic{Z(t)=j}] \\
\E[ Y(t) \indic{Z(t)=j}]
\end{pmatrix}
\end{align*}
With differential equation
\begin{align*}
\frac{d}{dt}\tilde{W}^j(t)=&
\sum_{g\neq j} \mu^{gj}(t) \tilde{W}^g(t) - \mu^{jg}(t) \tilde{W}^j(t)
\\
&+ \tilde{W}^j(t) \circ g_1(t,j,x,y)+p_{0j}(0,t) g_2(t,j)
\\
&+ \sum_{g \neq j} \mu^{gj}(t) \left( \tilde{W}^g(t) \circ h_1(t,g,j,x,y)+ p_{0g}(0,t) h_2(t,g,j) \right),
\\
\tilde{W}^j(0)=&\indic{Z(0)=j} \begin{pmatrix}
X(0)\\
Y(0)
\end{pmatrix},
\end{align*}
where $\circ$ denotes the Hadamard product (element-wise multiplication) and
\begin{gather*}
g_1(t,j,x,y)=\begin{pmatrix}
g_{x1}(t,j,y) \\
g_{y1}(t,j,x)
\end{pmatrix},
\qquad 
\quad
h_1(t,j,g,x,y)=\begin{pmatrix}
h_{x1}(t,j,g,y) \\
h_{y1}(t,j,g,x)
\end{pmatrix},
\\
g_2(t,j)=\begin{pmatrix}
g_{x2}(t,j,y) \\
g_{y2}(t,j,x)
\end{pmatrix},
\qquad 
\quad
h_2(t,j,g)=\begin{pmatrix}
h_{x2}(t,j,g,y) \\
h_{y2}(t,j,g,x)
\end{pmatrix}.
\end{gather*}
For 
\begin{align*}
g_{x1}(t,j,y)=&r^*(t)  +\delta_2^j(t)+\delta_4^j(t)y\\
&+\frac{b^j_2(t)}{V_2^{j*}(t)}
-\sum_{g \neq j} \rho_1^{jg}(t)
-\sum_{g \neq j} \left(\frac{b^{jg}_2(t)}{V_2^{j*}(t)} + \frac{V^{g*}_2(t)}{V^{j*}_2(t)}-1 \right) \mu^{jg}(t)
\\
g_{x2}(t,j,y)=& \delta_1^j(t)
+\delta_3^j(t)y
-b_1^j(t)
-\frac{V_1^{j*}(t)}{V_2^{j*}(t)}b^j_2(t)
-\sum_{g \neq j} \rho_2^{jg}(t)
\\
&-\sum_{g \neq j} \left( b_1^{jg}(t)-\frac{V_1^{j*}(t)}{V_2^{j*}(t)}b^{jg}_2(t) -\frac{V^{j*}_1(t)}{V^{j*}_2(t)}V^{g*}_2(t)+V^{g*}_1(t) \right) \mu^{jg}(t).
\end{align*}
\begin{align*}
h_x(t,j,g,x,y)=&\chi^{jg}(t,x) -x
\\
=&
V^{g*}_1(t) + \frac{x-V^{j*}_1(t)}{V^{j*}_2(t)}V^{g*}_2(t) -x
\\
=&
x
\underbrace{\left( \frac{V^{g*}_2(t)}{V^{j*}_2(t)}-1 \right)}_{h_{x1}(t,j,g,y)}+\underbrace{V^{g*}_1(t) - \frac{V^{j*}_1(t)V^{g*}_2(t)}{V^{j*}_2(t)}}_{h_{x2}(t,j,g,y)}.
\end{align*}
For $g_y$ we get
\begin{align*}
g_y(t,j,x,y)=&y \frac{dS(t)}{S(t)} - \delta^j(t,x,y) + (r(t)-r^*(t)) x + \sum_{g \neq j} \rho^{jg}(t,x) 
\\
=&y\frac{dS(t)}{S(t)} - \delta_1^j(t)-\delta_2^j(t)x-\delta_3^j(t)y-\delta_4^j(t)xy
\\
&+ (r(t)-r^*(t)) x
+\sum_{g \neq j} \rho^{jg}(t,x)
\\
=& y \underbrace{\left( \frac{dS(t)}{S(t)}-\delta_3^j(t)-\delta_4^j(t)x\right)
}_{g_{y1}(t,j,x)}\\
& \underbrace{+\sum_{g \neq j} \rho^{jg}(t,x)- \delta_1^j(t)-\delta_2^j(t)x
+ (r(t)-r^*(t)) x }_{g_{y2}(t,j,x)}
\end{align*}
Finally, as $h_y=0$ we have $h_{y1}=h_{y2}=0$.

\newpage
\bibliographystyle{plainnat}
\bibliography{BIBS}

\end{document}
