\documentclass[12pt]{article}
\usepackage[pdfstartview=FitH,hidelinks]{hyperref}
\usepackage[british]{babel}
\usepackage{a4,graphicx}
\usepackage[a4paper, hmargin={2.05cm, 2.05cm}]{geometry} 
\usepackage{amsmath,amssymb,amsthm,mathtools}
\usepackage{anyfontsize}
\usepackage{bbm}
%\usepackage{xypic}
\usepackage[latin1,utf8]{inputenc}
\usepackage{marvosym}
\usepackage{etoolbox}
\usepackage{relsize}
\usepackage{needspace}
\usepackage{nameref}
\usepackage{dsfont}
%\usepackage{thmtools}
%\usepackage{ntheorem}
%\newtheorem{lho}{Sætning}
\usepackage{filecontents}
\usepackage{tikz}
\usetikzlibrary{arrows.meta,positioning,calc}
\usetikzlibrary{matrix}
\usepackage{empheq}

\newcommand*\widefbox[1]{\fbox{\hspace{2em}#1\hspace{2em}}}
\newcommand{\N}{\mathbb{N}}
\newcommand{\R}{\mathbb{R}}
\newcommand{\Q}{\mathbb{Q}}
\newcommand{\Z}{\mathbb{Z}}
\newcommand{\B}{\mathcal{B}}
\newcommand{\F}{\mathcal{F}}
\newcommand{\E}{\text{E}}
\newcommand{\cov}{\text{cov}}
\newcommand{\indic}[1]{\mathds{1}_{ \{ #1 \} }}
\newcommand{\unv}[1]{\mathds{1}_{  #1  }}
\newcommand\ddfrac[2]{\frac{\displaystyle #1}{\displaystyle #2}}
\newcommand{\noin}{\noindent}
\newcommand{\Var}{\text{Var}}
\renewcommand{\P}{\text{P}}
\renewcommand{\baselinestretch}{1.25} 

\newcommand\independent{\protect\mathpalette{\protect\independenT}{\perp}}
\def\independenT#1#2{\mathrel{\rlap{$#1#2$}\mkern2mu{#1#2}}}
\font\tt=rm-lmtl10



\usepackage{caption}
\usepackage{listings,lstautogobble}
\usepackage{color}
\usepackage{float}
\usepackage{cprotect}
\usepackage[round, comma]{natbib}
\usepackage{csquotes}


\iffalse
\begin{filecontents*}{BIBS.bib}

@article{Djehiche,
issn = {01676687},
abstract = {We suggest a unified approach to claims reserving for life insurance policies with reserve-dependent payments driven by multi-state Markov chains. The associated prospective reserve is formulated as a recursive utility function using the framework of backward stochastic differential equations (BSDE). We show that the prospective reserve satisfies a nonlinear Thiele equation for Markovian BSDEs when the driver is a deterministic function of the reserve and the underlying Markov chain. Aggregation of prospective reserves for large and homogeneous insurance portfolios is considered through mean-field approximations. We show that the corresponding prospective reserve satisfies a BSDE of mean-field type and derive the associated nonlinear Thiele equation. [web URL: http://www.sciencedirect.com/science/article/pii/S0167668715300548]},
journal = {Insurance, Mathematics and Economics},
volume = {69},
publisher = {Elsevier Sequoia S.A.},
year = {2016},
title = {Nonlinear reserving in life insurance: Aggregation and mean-field approximation},
language = {eng},
address = {Amsterdam},
author = {Djehiche, Boualem and Löfdahl, Björn},
keywords = {Studies ; Life Insurance ; Insurance Claims ; Insurance Policies ; Differential Equations ; Stochastic Models ; Life and Health Insurance ; Experiment/Theoretical Treatment},
url = {http://search.proquest.com/docview/1806433652/},
}

@article{Norberg,
journal = {Scand. Actuar. J. 1},
year = {1991},
title = {Reserves in Life and Pension Insurance},
author = {Norberg, Ragnar},
pages={3-24}}

@book{Pardoux,
series = {Stochastic Modelling and Applied Probability},
volume = {69},
publisher = {Springer International Publishing},
isbn = {9783319057132},
year = {2014},
title = {Stochastic Differential Equations, Backward SDEs, Partial Differential Equations},
edition = {2014},
language = {eng},
address = {Cham},
author = {Pardoux, Etienne and Rascanu, Aurel},
keywords = {Mathematics ; Probability Theory and Stochastic Processes ; Partial Differential Equations ; Mathematics},
}

@article{THM_BUC,
issn = {03461238},
abstract = {Within the setup of a semi-Markov process in a finite state space, we consider a life insurance contract. First, without the modelling of policyholder behaviour, we show how to calculate the expected cash flow associated with future payments, and to that end we present a version of Kolmogorov's forward integro-differential equation. The semi-Markov model is then extended to include modelling of surrender and free policy behaviour, and the main result is a modification of Kolmogorov's forward integro-differential equation, such that the cash flow can be calculated without significantly more complexity than the cash flow without policyholder modelling. The result is also demonstrated for the traditional Markov case where there is no duration dependence, and numerical examples are studied.},
journal = {Scandinavian Actuarial Journal},
volume = {2015},
publisher = {Taylor & Francis Ltd.},
number = {8},
year = {2015},
title = {Cash flows and policyholder behaviour in the semi-Markov life insurance setup},
language = {eng},
address = {Stockholm},
author = {Buchardt, Kristian and Møller, Thomas and Schmidt, Kristian},
keywords = {Studies ; Cash Flow ; Markov Analysis ; Life Insurance ; Differential Equations ; Experiment/Theoretical Treatment ; Management Science/Operations Research ; Life & Health Insurance},
url = {http://search.proquest.com/docview/1718994569/},
}



@book{Liv2,
series = {International Series on Actuarial Science},
publisher = {Cambridge University Press},
isbn = {0521868777},
year = {2007},
title = {Market-valuation methods in life and pension insurance},
language = {eng},
address = {Cambridge},
author = {Møller, Thomas and Steffensen, Mogens},
keywords = {Insurance - Mathematics; Insurance, Life - Policies - Mathematics; Insurance, Pension trust guaranty - Mathematics; Life insurance policies - Mathematics; Pension trust guaranty insurance - Mathematics; Pension trusts - Mathematics; Økonomi, forsikring},
}

@article{Steffensen0,
issn = {01676687},
abstract = {The multi-state life insurance contract is reconsidered in a framework of securitization where insurance claims may be priced by the principle of no arbitrage. This way a generalized version of Thiele's differential equation is obtained for insurance contracts linked to indices, possibly marketed securities. The equation is exemplified by a traditional policy, a simple unit-linked policy and a half-dependent unit-linked policy.},
journal = {Insurance, Mathematics & Economics},
pages = {201--214},
volume = {27},
publisher = {Elsevier Sequoia S.A.},
number = {2},
year = {2000},
title = {A no arbitrage approach to Thiele's differential equation},
language = {eng},
address = {Amsterdam},
author = {Steffensen, Mogens},
keywords = {Arbitrage ; Securitization ; Market Prices ; Insurance Policies ; Stochastic Models ; Economic Theory ; Studies ; Economic Theory ; Insurance Industry ; Experimental/Theoretical},
url = {http://search.proquest.com/docview/208166412/},
}

@book{Steffensen1,
publisher = {Laboratory of Actuarial Mathematics, University of Copenhagen},
isbn = {8778344492},
year = {2001},
title = {On valuation and control in life and pension insurance},
language = {eng},
address = {Copenhagen},
author = {Steffensen, Mogens},
}



@article{NorbergB,
issn = {0949-2984},
abstract = {The issue of bonus in life insurance is considered in a model framework where the traditional set-up is extended by letting the experience basis (mortality, interest, etc.) be stochastic. A novel definition of the technical surplus on an insurance contract is proposed, and basic principles for its repayment as bonus are discussed. Making the experience basis an endogenous part of the model opens possibilities of model-based prognostication of future bonuses. Numerical illustrations are provided.},
journal = {Finance and Stochastics},
pages = {373--390},
volume = {3},
publisher = {Springer-Verlag},
number = {4},
year = {1999},
title = {A theory of bonus in life insurance},
language = {eng},
address = {Berlin Heidelberg},
author = {Norberg, Ragnar},
keywords = {Key words: Safety margins, prospective reserves, retrospective reserves, stochastic interest, stochastic mortality, counting processes ; JEL Classification:G22, G23 ; Mathematics Subject Classification (1991):60J27, 62P05},
}



\end{filecontents*}

\fi



\begin{document}

\subsection*{Stuff to fix}
\begin{itemize}
\item Jeg kommer med flere påstande om industrien som jeg ikke er sikker på har hold i virkeligheden. fx at forsikringsselskaber ikke har bekymret sig for om deres udlodningsstrategier er fair på policeniveau.
\end{itemize}
\section*{Introduction}

With-profit insurance contracts are to this day one of the most popular life insurance contracts. They arose as a natural way to distribute the systematic surplus that develops due to the prudent assumptions on which the contract is made, and in the recent years a lot of attention has been aimed at how this distribution of surplus should be conducted fairly. In this regard, fairness means that the value of contributions to the surplus correspond to the value of the bonus received. If one is content with fairness on the portfolio level, it is sufficient to require that the collective surplus should be zero when all contracts are terminated. However, if we wish to distribute fairly on the policy level, more information about the development of each individual policy is needed. In this paper we derive a retrospective differential equation for the savings account, which aids in determining how much each individual policy has contributed to the collective surplus. Apart from providing a tool for fair distribution of surplus, the method provides insight on the financial risks carried by the insurance companies.\\
The savings account, henceforth referred to as savings, differs from the reserves that classically are used to asses insurance liabilities, for in particular one reason: it depends on the entire history of the policy. This poses a problem, as the Markov property that usually saves us from the hassle of dealing with the past, no longer can be directly applied. Furthermore, the prospective reserves are used to find the liabilities of today, while the savings of today already are known. the notion of bonus has been studied See chapter 6 of \citet{Liv2}... [but as insurance companies have been satisfied with fairness on portfolio level, there has been little effort invested in the surplus generation on the individual policy]. Much of the theory concerning surplus has not taken into account that dividends usually are spent on increasing the policies benefits. This increase in benefits has an impact on the dynamics of the savings, and several of the  [\citep{NorbergB} concerned with the size of the surplus belonging to each policy, we are interested in the case where surplus is used to buy more insurance.]. Bonus emerges due to the difference between the first and second order basis. The first order basis


\subsection*{Setup}
We consider the classic multi-state life insurance setup, comprised of a state process $Z$ denoting the state of the policy in a finite state space $\mathcal{J}=\{0,1,...,J\}$. 
\begin{itemize}
\item Definition of $X$ and $Y$
\item Deterministic second order basis, and discussion regarding simulation.
\item remark on continuity of $b^i(t)$
\item model limitations.
\end{itemize}
The dynamics of $X$ are assumed to be affine
\begin{align*}
dX(s)=&X(s)g_1(s,Z(s))ds+g_2(s,Z(s))ds\\
&+\sum_{h\neq Z(s-)} \left( X(s-)h_1(s,Z(s-),h)+ h_2(s,Z(s-),h)\right) dN^h(s).
\end{align*}

In practice, the surplus account is shared among policyholders. The increased benefits are also guaranteed, which wi do not consider.

\subsection*{State-0 model}
We consider a simple model where the expected future savings are described by an easily derived differential equation. The model consists of $n$ inactive states where there are no payments, and one active state with a non-zero savings-dependent payment process $b$. On transition to one of the inactive states, the savings are nullified. We need not specify what happens to the savings on a transition - they may be paid out to the customer or the insurance company, or any combination of the two - the only important requirement is that the savings are zero in all inactive states. As there are no payments in the inactive states, $\chi^{0j}(t,x)=0$ for all inactive states $j$, implying that $h_x(t,0,j,x,y)=-x$. While the policy is in the active state, it contributes and receives dividend from a surplus account, $Y$. The survival model with and without surrender options are special cases of this model. Denote by $0$ the active state. The dynamics of $X$ are
\begin{align*}
dX(s)=& \indic{Z(s-)=0}g(s,0,X(s))ds - \sum_{h=1}^n X(s-)dN^h(s).
\end{align*}
Note that there are no risk premiums, as there is no risk for the insurance company company related to the transitions of the policy. Therefore, the only risk carried by the insurance company, relates to the interest of the savings account.



Remark: If the benefits are identical after age 65, the states 0,1,3 and 4 can be lumped, as well as 2,5 and 6, thus creating a survival model. If the dynamics in two states are identical, they can be viewed as one. Life annuity at age 65.


\def\PlA{(0,0)}
\begin{figure}[H]
\begin{center}
\begin{tikzpicture}
\begin{scope}[every node/.style={rectangle,thick,draw,inner sep=10pt,minimum width=2cm,minimum height = 1cm},rounded corners=1mm]
    \node (0) at \PlA {0, alive};
    \node (1) at ($(0) + (4.5,0)$) {1, dead};
\end{scope}

\begin{scope}[>={Stealth[black]},
              every node/.style={fill=white,circle,scale=0.9},
              every edge/.style={draw=black,very thick}]
    \path [->] 	(0) edge [bend left=0] node {$\mu$} (1);
\end{scope}
\end{tikzpicture}
\label{fig:1}
\caption{Life-Death model}
\end{center}
\end{figure}



\section*{State-Wise Probability Weighted Reserve}
Define
\begin{align*}
\tilde{X}^j(t):=\E_{Z(0)}[X(t)\indic{Z(t)=j}]
\end{align*}
and note that
\begin{align}
\E_{Z(0)}[X(t)\indic{Z(t)=j}]= \E_{Z(0)}[X(t)|Z(t)=j]p_{Z(0),j}(0,t), \label{eq:1}
\end{align}
by the definition of conditional expectation. We can think of $\tilde{X}^j$ as the probability weighted state-wise reserves. The relation between $\tilde{X}^j$ and $\E[X(t)]$ is
\begin{align*}
\E_{Z(0)}[X(t)] =& \E_{Z(0)}[\E_{Z(0)} [ X(t)|Z(t)]] 
\\
=&
\E_{Z(0)} \left[ \sum_{j\in \mathcal{J}} \indic{Z(t)=j} \E_{Z(0)} [ X(t)|Z(t)=j] \right]
\\
=&
\E_{Z(0)} \left[ \sum_{j\in \mathcal{J}} \indic{Z(t)=j} \frac{\E_{Z(0)}[X(t)\indic{Z(t)=j}]}{p_{0j}(0,t)} \right]
\\
=&
\sum_{j\in \mathcal{J}} p_{0j}(0,t) \frac{ \tilde{X}^j}{p_{0j}(0,t)}
\\
=&
\sum_{j\in \mathcal{J}} \tilde{X}^j.
\end{align*}




\subsection*{Life-Death-Surrender}
\def\PlA{(0,0)}
\begin{center}
\begin{tikzpicture}
\begin{scope}[every node/.style={rectangle,thick,draw,inner sep=10pt,minimum width=2cm,minimum height = 1cm},rounded corners=1mm]
    \node (0) at \PlA {0, alive};
    \node (1) at ($(0.center) + (4.9,0)$) {1, dead};
    \node (2) at ($(0.center) + (-4.5,0)$) {2, surrender};    
\end{scope}

\begin{scope}[>={Stealth[black]},
              every node/.style={fill=white,circle,scale=0.9},
              every edge/.style={draw=black,very thick}]
    \path [->] 	(0) edge [bend left=0] node {$\mu^{01}$} (1);
    \path [->] 	(0) edge [bend left=0] node {$\mu^{02}$} (2);
\end{scope}
\end{tikzpicture}
\end{center}



\subsection*{Life-Death-Surrender With Free Policy}
\def\PlA{(0,0)}
\def\PlB{(0,-3)}
\begin{center}
\begin{tikzpicture}
\begin{scope}[every node/.style={rectangle,thick,draw,inner sep=10pt,minimum width=2cm,minimum height = 1cm},rounded corners=1mm,align=center]
    \node (0) at \PlA {0, alive};
    \node (1) at ($(0.center) + (4.5,0)$) {1, dead};
    \node (2) at ($(0.center) + (-4.5,0)$) {2, surrender}; 
    \node (3) at \PlB {3, alive\\ free policy};
    \node (4) at ($(3.center) + (4.5,0)$) {4, dead \\ free policy};
    \node (5) at ($(3.center) + (-4.5,0)$) {5, surrender \\ free policy};    
\end{scope}

\begin{scope}[>={Stealth[black]},
              every node/.style={fill=white,circle,scale=0.9},
              every edge/.style={draw=black,very thick}]
    \path [->] 	(0) edge [bend left=0] node {$\mu^{01}$} (1);
    \path [->] 	(0) edge [bend left=0] node {$\mu^{02}$} (2);
    \path [->] 	(0) edge [bend left=0] node {$\mu^{03}$} (3);
    \path [->] 	(3) edge [bend left=0] node {$\mu^{34}$} (4);
    \path [->] 	(3) edge [bend left=0] node {$\mu^{35	}$} (5);
\end{scope}
\end{tikzpicture}
\end{center}

\subsection*{Thoughts}
\begin{itemize}
\item With-profit insurance! Expected reserve including accumulation of dividends.
\item Same build-up as \citet{THM_BUC}. Hierarchical examples $\Rightarrow$ general transient.
\item Refer to \citet{Norberg}
\begin{itemize}
\item Introduction and motivation - stochastic reserve, Monte Carlo method. A little comment on the fact that the problem is still hard to solve.
\item Life-death (simple analytic solution).
\item Life-death free policy (how to deal with extra states).
\item General model without duration.
\item Life-death-surrender free policy, including discussion of free policy factor.
\item Lost all trick works.
\item General model with duration dependence.
\item Inclusion of surplus. Use independence when dividend is assigned on discrete points in time.
\end{itemize}
\item Deterministic intensities.
\item Market dependent intensities - allowed when directly dependent on the market, making them deterministic.
\item We are only concerned with the reserve.
\item Maybe we should use a different wording? \textbf{Savings}/stash/backlog/accumulation/hoard/reservoir instead of reserve, to distinguish between the Danish words for "reserve" and "depot"
\end{itemize}


\newpage
\bibliographystyle{plainnat}
\bibliography{BIBS}

\end{document}
